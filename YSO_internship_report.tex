
\documentclass{article}
\usepackage{amsmath}
\usepackage{graphicx}
\usepackage{hyperref}
\usepackage{listings}
\usepackage{color}

\definecolor{codegray}{rgb}{0.5,0.5,0.5}
\definecolor{codegreen}{rgb}{0,0.6,0}
\definecolor{codepurple}{rgb}{0.58,0,0.82}
\definecolor{backcolour}{rgb}{0.95,0.95,0.92}

\lstdefinestyle{mystyle}{
    backgroundcolor=\color{backcolour},   
    commentstyle=\color{codegreen},
    keywordstyle=\color{magenta},
    numberstyle=\tiny\color{codegray},
    stringstyle=\color{codepurple},
    basicstyle=\ttfamily\footnotesize,
    breakatwhitespace=false,         
    breaklines=true,                 
    captionpos=b,                    
    keepspaces=true,                 
    numbers=left,                    
    numbersep=5pt,                  
    showspaces=false,                
    showstringspaces=false,
    showtabs=false,                  
    tabsize=2
}

\lstset{style=mystyle}

\title{Variability of Young Stellar Objects — Internship Report}
\author{Chiara Virzì}
\date{April 2025}

\begin{document}

\maketitle

\section{Introduction}
This report outlines the work I conducted during my internship under the supervision of Dr. Rosaria Bonito at the Osservatorio Astronomico di Palermo. The focus of the internship was the analysis of variability in Young Stellar Objects (YSOs), using both photometric and spectroscopic data, in preparation for the upcoming Rubin LSST survey.

The aim of this project is to study spectral variability in Young Stellar Objects (YSOs), with a focus on the H$\alpha$ emission line, across timescales of several months. As a Rubin LSST data rights holder, I developed methods to identify accretion signatures in archival spectra and correlated these features with photometric variability. The H$\alpha$ line serves as a key diagnostic for accretion processes, mass-loss activity, and the evolutionary stages of pre-main-sequence stars.

\section{Young Stellar Object Science}

YSOs are stars in their early formation stages, still undergoing contraction and typically surrounded by disks of gas and dust. These stars are known for their strong variability, driven by accretion, magnetic activity, and interactions with surrounding circumstellar material.

\subsection{YSO Evolutionary Classes}

YSOs are classified based on their spectral energy distributions (SEDs) and infrared excess:
\begin{itemize}
    \item \textbf{Class 0}: Deeply embedded protostars, observed mainly in the submillimeter range.
    \item \textbf{Class I}: Protostars with infalling envelopes, showing strong infrared excess and outflows.
    \item \textbf{Class II}: Classical T Tauri Stars (CTTS) with developed accretion disks and active mass transfer.
    \item \textbf{Class III}: Weak-lined T Tauri Stars (WTTS) with little or no disk and minimal accretion signatures.
\end{itemize}

Understanding Class II and III sources is crucial for studying late-stage star formation and early disk evolution — key elements for understanding planet formation.\footnote{Reference [4]}

\subsection{Accretion Bursts}

Accretion bursts are episodic events in which disk material rapidly accretes onto the stellar surface, temporarily increasing brightness and temperature. Localized hot spots dominate the UV/blue spectrum during these events.

Different photometric bands respond uniquely:
\begin{itemize}
    \item \textbf{u/g bands}: Trace hot-spot and shock emission.
    \item \textbf{i/z bands}: Sensitive to extinction changes and disk occultation.
\end{itemize}

\subsubsection{Short-Term Variability}
Occurs over hours to days, driven by magnetic flares, unstable accretion streams, or jet activity.

\subsubsection{Long-Term Variability}
Occurs over months to years, including EXor-type (months-long) and FUor-type (decades-long) outbursts. In particular, EXors — characterized by episodic accretion outbursts lasting months to years — are still poorly understood due to the rarity of available samples.

\subsection{Light Curves}

Light curves record brightness versus time, revealing:
\begin{itemize}
    \item \textbf{Periodic variability} from rotation or disk warps.
    \item \textbf{Aperiodic variability} from changing accretion or extinction.
\end{itemize}

In this project, ZTF light curves were cross-matched with H$\alpha$ spectroscopic diagnostics to identify correlated variability.\footnote{Reference [12]}

\subsection{Spectroscopy and the H$\alpha$ Line}

The H$\alpha$ line at 6562.8 Å provides direct insight into accretion and wind phenomena.
\begin{itemize}
    \item \textbf{Broad/asymmetric profiles} indicate magnetospheric infall or outflow.
    \item \textbf{Narrow/symmetric profiles} indicate chromospheric emission.
\end{itemize}

The Full Width at Zero Intensity (FWZI) measures the total emission-line width — from continuum departure to return — and serves as a proxy for maximum gas velocities in the line-forming region.

\section{Code}

\subsection{Import Necessary Libraries}

\begin{lstlisting}[language=Python]
# Import necessary libraries
import os
import numpy as np
import matplotlib.pyplot as plt
from astropy.io import fits
\end{lstlisting}

\subsection{Function to Read FITS Files}

\begin{lstlisting}[language=Python]
# Function to read FITS files
def read_fits(file):
    with fits.open(file) as hdul:
        header = hdul[0].header
        data = hdul[0].data

    crval1 = header['CRVAL1']
    crpix1 = header['CRPIX1']
    cdelt1 = header['CDELT1']
    npix   = header['NAXIS1']

    # Wavelength (in Å)
    x = crval1 + (np.arange(npix) - (crpix1 - 1)) * cdelt1
    return x, data
\end{lstlisting}

\end{document}
